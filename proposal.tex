\documentclass[11pt]{article}

\author{Hodai Goldman (Hodaig@cs.bgu.ac.il) \\
Department of Computer Science, \\
Ben-Gurion University of the Negev, Beer Sheva, Israel \\
Supervisor: Dr. Gera Weiss}
\date{\today}
\title{Research Proposal: Computational resource management of multi channel controller}

\begin{document}
\begin{titlepage}
\maketitle
\end{titlepage}




\section{Background and Problem Formulation}
\label{sec:problem}
Today’s computer power allows for consolidation of controllers where a single computer can regulate many control loops, each with its varying needs of computation resources.
This brings two research challenges that we intend to attack in the proposed thesis:
\begin{itemize}
	\item How to schedule control tasks in order to achieve good performance in terms of control measures (overshoot, convergence speed, etc.)?
	\item What is a good interface for co-design of scheduling and control?
\end{itemize}

While it is possible to build control systems using standard operating systems, either real-time or desktop with static or with dynamic scheduling schemes, there is an agreed opinion in the control community that these do not serve well for the purpose outlined above~\cite{?}. Specifically, desktop type operating systems (Windows, Linux, etc.) schedule for computational efficiency, but do not allow for control performance guarantees of the individual loops. On the other hand, real-time operating systems sacrifice some efficiency for timing predictability, but they do relate timing information with control performance. When using such operating systems for control engineers usually apply controllers that work in a fixed periodic manner so that control behavior becomes deterministic and control performance can be guaranteed. This is not efficient because resources can be better utilized if controllers act at higher frequencies when only when needed. 
In this work we will develop methods to combine the efficiency of desktop operating systems with the predictability of real-time operating systems in a way that is more suitable for control systems then periods and deadlines

%Although scheduling is a well researched area, covered in many research papers and books, existing scheduling theory is not yet fully adopted to work in the context of control systems~\cite{?}.
%In the engineering practice, typically, scheduling mechanisms can be characterized as one of two types:
%\begin{description}
%	\item[Dynamic Schedule:]{
%		a commonly uses scheduling method in desktop operation systems (eg. Windows, Linux) that aims at efficiency of CPU usage.
%	}
%	
%	\item[Real-Time schedule:]{
%		such as embedded systems for automatic control (for example robotics), 
%		scheduling is guided by specifications of \textbf{periods} and \textbf{deadlines}, the scheduler must ensure that processes can meet deadlines. this limitations are for keeping the system stable but usually not efficient.
%	}
%\end{description}
%in this thesis we will suggest a technique to combine the efficiency of desktop operating systems with the predictability of real-time operating systems,
%in a way that is more suitable for control systems then periods and deadlines (see section TODO ).

The control loops that we are analyzing are here of the form shown in Figure~\ref{fig:control loop}. A physical plant (controlled system) is connected via sensors and actuators... % Gera: explain the figure

The current state of the art is that control engineers design control tasks as periodic computations then they specify the required periodic frequency for the task and software engineers design a scheduler that ensures that the periodic frequency requirements are met usually using pre-computed knowledge of the expected (maximum) duration of the tasks.
We claim that for control systems we can achieve better performance by using richer and more flexible set of requirements for the tasks. Specifically, we will develop tools with which the control engineers can specify in a natural way features of their control loop that the scheduler will use to allow for dynamic schedules that guarantee required control performance. 


%In this thesis we suggest a novel technique where the control loop is composed (as usual) from control tasks and state estimation tasks. 
%And the requirement of the periodical requirements of the control tasks dynamically change and depend on the level of the environmental noise that is expressed by the accuracy of the state estimation (retrieve from estimation loops) and the stability of the system (retrieve from the control loop).

\section{Case study: Vision based controllers for drones}



\section{Research Plan}





% Gera: Use bibtex 
%This research continues/completes the work of Merav Bukra~[1] on scheduling computations in embedded control systems.
%In her thesis Merav presented an approach and a proof-of-concept tool for scheduling computation resources in control software. The tool was good enough for demonstrating the concept with simulations, but it did not integrate with any real-life system. In this research we will develop a resource scheduler for real-life control model (see~\ref{sec:contmode}),
%and upgrade the control software (APM) of a well-known quad-copter by replacing it's RT-scheduler with our new scheduler.
%We expect an improvement in CPU utilization that will allow improvements of stability of the drone.

\subsection{the controller control system framework}
our proposal.

TODO - add the figure from the poster and few words of this model
TODO - maybe add \%CPU arrow from scheduler to control law 

\subsubsection{Sensors (Computer Vision Based Sensor)}
\subsubsection{State Estimator}
\subsubsection{Control Tasks}
\subsubsection{Automata Based Scheduler}

\subsection{experimental environment}
\subsubsection{drones}
\subsubsection{Ardu-Pilot-Mega as base controller software}

\section{Preliminary Results}
Blablabla said Nobody ~\cite{Merav}.
Blablabla said Nobody ~\cite{APM}.
Blablabla said Nobody ~\cite{RTComposer}.


\bibliography{proposal}{}
\bibliographystyle{plain}


\end{document}}